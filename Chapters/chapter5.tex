\chapter{تحلیل آماری نتایج بهینه‌سازی و استخراج محدوده پارامترهای بهینه}

\section{مقدمه و بیان مسئله}

\subsection{چالش عدم تکرارپذیری در بهینه‌سازی عددی}

یکی از چالش‌های اساسی در بهینه‌سازی عددی پارامترهای جاذب‌های دینامیکی ارتعاش (\lr{Dynamic Vibration Absorbers} یا \lr{DVAs})، مسئله عدم تکرارپذیری نتایج است. به عبارت دیگر، اجرای الگوریتم‌های بهینه‌سازی بر روی یک مسئله مشخص، هر بار ممکن است به پارامترهای متفاوتی منجر شود که همگی ادعای بهینه بودن دارند. این پدیده ناشی از ماهیت تصادفی و غیرقطعی بسیاری از الگوریتم‌های بهینه‌سازی مدرن است.

فرض کنید مسئله بهینه‌سازی را به صورت زیر تعریف کنیم:

\begin{equation}
\min_{\theta} \; f(\theta) \quad \text{به طوری که} \quad \theta \in \Theta
\end{equation}

که در آن $\theta$ بردار پارامترهای طراحی جاذب، $f(\theta)$ تابع هدف (مانند معیار \lr{Peak-Slope})، و $\Theta$ فضای پارامتری مجاز است. در عمل، اجرای چندین بار الگوریتم بهینه‌سازی ممکن است به مجموعه‌های مختلفی از پارامترهای بهینه منجر شود:

\begin{equation}
\theta_1^*, \theta_2^*, \dots, \theta_n^* \quad \text{که همگی} \quad f(\theta_i^*) \approx f_{\mtext{opt}}
\end{equation}

این عدم قطعیت به دلایل متعددی رخ می‌دهد:
\begin{enumerate}
    \item \textbf{ماهیت تصادفی الگوریتم‌ها}: الگوریتم‌های مبتنی بر جمعیت مانند الگوریتم ژنتیک دارای مؤلفه‌های تصادفی هستند
    \item \textbf{چندراه‌حلی بودن مسئله}: فضای طراحی ممکن است دارای چندین مینیمم محلی باشد
    \item \textbf{وابستگی به شرایط اولیه}: نتایج ممکن است به شرایط اولیه الگوریتم حساس باشند
    \item \textbf{محدودیت‌های عددی}: دقت محاسباتی و اندازه جمعیت بر نتایج تأثیرگذار است
\end{enumerate}

\subsection{نیاز به محدوده پارامترهای بهینه در کاربردهای عملی}

\subsubsection{محدودیت‌های سیستم‌های گسسته}
در کاربردهای عملی، سیستم‌های مهندسی اغلب دارای محدودیت‌های ذاتی در انتخاب پارامترهای طراحی هستند. این محدودیت‌ها ممکن است ناشی از ماهیت گسسته پارامترها، در دسترس بودن اجزای استاندارد در بازار، یا محدودیت‌های تأمین‌کنندگان خاص باشد. به عنوان مثال، در طراحی جاذب‌های دینامیکی ارتعاش، پارامترهایی نظیر نسبت جرم، نسبت سختی، و نسبت میرایی معمولاً محدود به مقادیر استاندارد موجود در بازار هستند و نمی‌توان آنها را به صورت پیوسته و دلخواه انتخاب کرد. علاوه بر این، ملاحظات اقتصادی، فنی، و عملیاتی نیز بر انتخاب این پارامترها تأثیر می‌گذارند. در نتیجه، به جای تکیه بر یک مجموعه پارامترهای نقطه‌ای و دقیق که ممکن است در عمل قابل دستیابی نباشد، نیاز به تعریف محدوده‌هایی از پارامترهای قابل قبول داریم که همگی عملکرد مطلوب و قابل قبولی ارائه دهند و در عین حال با محدودیت‌های عملی سازگار باشند.

\subsubsection{مزایای رویکرد محدوده‌ای}

استفاده از محدوده پارامترهای بهینه مزایای متعددی دارد. از جمله مهم‌ترین این مزایا می‌توان به \textbf{انعطاف‌پذیری طراحی} اشاره کرد که امکان انتخاب از میان گزینه‌های متعدد را فراهم می‌آورد و به طراح اجازه می‌دهد تا بر اساس شرایط خاص پروژه، بهترین گزینه را انتخاب کند. همچنین، این رویکرد \textbf{قابلیت اطمینان} بالایی دارد زیرا عدم قطعیت‌های عملیاتی و تغییرات احتمالی در شرایط کاری را پوشش می‌دهد. از منظر عملی، محدوده پارامترهای بهینه \textbf{سهولت پیاده‌سازی} بیشتری ارائه می‌دهد چرا که با محدودیت‌های تأمین و ساخت سازگار است و نیازی به دستیابی به مقادیر دقیق و خاص نیست. در نهایت، این روش منجر به \textbf{بهینه‌سازی هزینه} می‌شود زیرا نیاز به دقت بالا در ساخت و تولید را کاهش داده و امکان استفاده از قطعات و مواد استاندارد موجود در بازار را فراهم می‌آورد.

\section{روش‌شناسی تحلیل آماری نتایج}

\subsection{چارچوب کلی تحلیل}

این فصل روشی جامع برای تحلیل آماری نتایج بهینه‌سازی ارائه می‌دهد که مستقل از روش بهینه‌سازی مورد استفاده است. رویکرد ارائه‌شده در نرم‌افزار \lr{DeVana} پیاده‌سازی شده و شامل مراحل زیر است:

\begin{enumerate}
    \item \textbf{جمع‌آوری داده‌ها}: اجرای چندین بار بهینه‌سازی و جمع‌آوری مجموعه پارامترهای بهینه
    \item \textbf{تجزیه آماری}: تحلیل توزیع پارامترها در فضای نتایج
    \item \textbf{استخراج محدوده‌ها}: استفاده از معیارهای آماری مختلف برای تعیین محدوده‌های قابل قبول
    \item \textbf{انتخاب محدوده نهایی}: انتخاب بهترین محدوده بر اساس معیارهای کیفیت
\end{enumerate}

\subsection{معیار آماری پیشنهادی}

\subsubsection{روش \lr{Top 10\% Narrow Q47.5-Q52.5}}

این روش پیشرفته‌ترین و دقیق‌ترین رویکرد است که در این پژوهش به‌عنوان روش اصلی پیشنهاد می‌شود. مراحل این روش عبارتند از:

\paragraph{مرحله ۱: انتخاب بهترین نتایج}
ابتدا ۱۰\% بهترین نتایج (با کمترین مقدار تابع هدف) انتخاب می‌شوند:

\begin{equation}
S_{10\%} = \{\theta^*_j \mid f(\theta^*_j) \leq Q_{10\%}(\{f(\theta^*_k)\})\}
\end{equation}

\paragraph{مرحله ۲: محاسبه محدوده باریک مرکزی}
از این زیرمجموعه، محدوده باریکی در مرکز توزیع انتخاب می‌شود (از صدک ۴۷.۵ تا ۵۲.۵):

\begin{equation}
\theta_{\mtext{Top10\%}} = [Q_{47.5}(S_{10\%}), Q_{52.5}(S_{10\%})]
\end{equation}

\paragraph{ویژگی‌های منحصربه‌فرد این روش}

این روش دارای چندین ویژگی منحصربه‌فرد است که آن را برای کاربردهای مختلف مناسب می‌سازد. \textbf{تمرکز بر بهترین نتایج} یکی از مهم‌ترین ویژگی‌های این روش است، به‌طوری‌که تنها بهترین ۱۰ درصد نتایج حاصل از فرآیند بهینه‌سازی مورد بررسی قرار می‌گیرند. این انتخاب گزینشی باعث می‌شود تا کیفیت محدوده‌های نهایی به‌طور قابل‌توجهی بهبود یابد. همچنین، \textbf{محدوده باریک مرکزی} که بر اساس تمرکز بر مرکز توزیع طراحی شده، دقت بالایی را در تعیین پارامترهای بهینه فراهم می‌آورد. \textbf{قابلیت اطمینان بالا} این روش نیز از مزایای مهم آن محسوب می‌شود، زیرا در برابر نتایج پرت و غیرعادی مقاومت نشان داده و پایداری مناسبی دارد. در نهایت، \textbf{انعطاف‌پذیری عملی} این رویکرد با ارائه محدوده‌های کوچک‌تر و قابل‌کنترل، امکان استفاده در کاربردهای گسسته و سیستم‌های با محدودیت‌های عملی را فراهم می‌آورد.

\section{الگوریتم طراحی نهایی}

\subsection{الگوریتم اصلی استخراج محدوده‌ها}

الگوریتم اصلی برای استخراج محدوده پارامترهای بهینه به شرح زیر است:

\textbf{ورودی:} مجموعه نتایج بهینه‌سازی $\Theta^* = \{\theta_1^*, \dots, \theta_n^*\}$، تابع هدف $f$

\textbf{خروجی:} محدوده پارامترهای توصیه‌شده $\Theta_{\mtext{recommended}}$ برای هر پارامتر

\textbf{الگوریتم:}

\begin{enumerate}
    \item \textbf{برای هر پارامتر} $\theta_i \in \Theta$:
    \begin{enumerate}
        \item استخراج مقادیر پارامتر: $\theta_i^* = \{\theta_{i,1}^*, \dots, \theta_{i,n}^*\}$
        \item محاسبه محدوده پیشنهادی با روش \lr{Top} 10\%:
        \begin{itemize}
            \item انتخاب بهترین ۱۰\% نتایج: $S_{10\%} = \{\theta^*_j \mid f(\theta^*_j) \leq Q_{10\%}(\{f(\theta^*_k)\})\}$
            \item محاسبه محدوده نهایی: $\theta_{\mtext{Top10\%}} = [Q_{47.5}(S_{10\%}), Q_{52.5}(S_{10\%})]$
        \end{itemize}
    \end{enumerate}

    \item \textbf{انتخاب روش نهایی:} استفاده از روش \lr{Top 10\% Narrow Q47.5-Q52.5} به عنوان روش پیش‌فرض

    \item \textbf{اعمال محدودیت‌های عملی:} اعمال محدودیت‌های تأمین و ساخت بر محدوده‌ها

    \item \textbf{بازگشت:} $\Theta_{\mtext{recommended}}$
\end{enumerate}

\subsection{ملاحظات طراحی و ایمنی}

\subsubsection{تفاوت تحمل بهینه‌سازی و تحمل واقعی}

در طراحی عملی، بین تحمل مورد نیاز برای بهینه‌سازی و تحمل واقعی سیستم تفاوت وجود دارد. این تفاوت به دلایل زیر است:

\begin{enumerate}
    \item \textbf{عدم قطعیت مدل‌سازی}: مدل‌های ریاضی تقریبی هستند
    \item \textbf{تفاوت‌های ساخت}: پارامترهای واقعی با پارامترهای طراحی متفاوت هستند
    \item \textbf{شرایط عملیاتی}: تغییرات دما، بار، و سایر عوامل محیطی
    \item \textbf{ایمنی طراحی}: نیاز به حاشیه ایمنی برای اطمینان از عملکرد قابل قبول
\end{enumerate}

\subsubsection{ضریب ایمنی پیشنهادی}

برای اطمینان از عملکرد قابل قبول در شرایط واقعی، پیشنهاد می‌کنیم:

\begin{equation}
\text{تحمل واقعی} = \text{تحمل بهینه‌سازی} \times \text{ضریب ایمنی}
\end{equation}

که در آن ضریب ایمنی معمولاً بین ۲ تا ۴ انتخاب می‌شود، بسته به حساسیت کاربرد.

\section{مطالعه ی موردی}
\subsection{پارامترهای ورودی مطالعه موردی}

در این بخش، پارامترهای ورودی مورد استفاده در مطالعه موردی ارائه می‌شود. این پارامترها شامل مشخصات سیستم اصلی، میراگر دینامیکی، اهداف بهینه‌سازی، و تنظیمات الگوریتم ژنتیک است.

\subsubsection{پارامترهای سیستم اصلی}

پارامترهای سیستم اصلی در جدول \ref{tab:main_system_params} ارائه شده است:

\begin{table}[h!]
\centering
\caption{پارامترهای سیستم اصلی}
\label{tab:main_system_params}
\begin{tabular}{|c|c|c|}
\hline
\textbf{پارامتر} & \textbf{مقدار} & \textbf{واحد} \\
\hline
$\mu$ & 1.0 & - \\
\hline
$\lambda_1$ & 1.0 & - \\
\hline
$\lambda_2$ & 1.0 & - \\
\hline
$\lambda_3$ & 0.5 & - \\
\hline
$\lambda_4$ & 0.5 & - \\
\hline
$\lambda_5$ & 0.5 & - \\
\hline
$\nu_1$ تا $\nu_5$ & 0.75 & - \\
\hline
$a_{low}$ & 0.05 & - \\
\hline
$a_{up}$ & 0.05 & - \\
\hline
$f_1$ & 100.0 & Hz \\
\hline
$f_2$ & 100.0 & Hz \\
\hline
$\omega_{dc}$ & 100.0 & rad/s \\
\hline
$\zeta_{dc}$ & 0.01 & - \\
\hline
\end{tabular}
\end{table}

\subsubsection{پارامترهای میراگر دینامیکی}

پارامترهای اولیه میراگر دینامیکی در جدول \ref{tab:dva_params} نشان داده شده است:

\begin{table}[h!]
\centering
\caption{پارامترهای اولیه میراگر دینامیکی}
\label{tab:dva_params}
\begin{tabular}{ccc}
\hline
\textbf{پارامتر} & \textbf{تعداد} & \textbf{مقدار اولیه} \\
\hline
$\beta$ & $15$ & $0.0$ \\
$\lambda$ & $15$ & $0.0$ \\
$\mu$ & $3$ & $0.0$ \\
$\nu$ & $15$ & $0.0$ \\
\hline
\end{tabular}
\end{table}

\subsubsection{تنظیمات فرکانسی}

پارامترهای مربوط به تحلیل فرکانسی در جدول \ref{tab:frequency_params} ارائه شده است:

\begin{table}[h!]
\centering
\caption{تنظیمات تحلیل فرکانسی}
\label{tab:frequency_params}
\begin{tabular}{ccc}
\hline
\textbf{پارامتر} & \textbf{مقدار} & \textbf{واحد} \\
\hline
$\omega_{start}$ & $0.0$ & rad/s \\
$\omega_{end}$ & $350.0$ & rad/s \\
تعداد نقاط & $350$ & - \\
روش درون‌یابی & none & - \\
نقاط درون‌یابی & $1000$ & - \\
\hline
\end{tabular}
\end{table}

\subsubsection{تنظیمات الگوریتم ژنتیک}

پارامترهای الگوریتم ژنتیک مورد استفاده در جدول \ref{tab:ga_params} نشان داده شده است:

\begin{table}[h!]
\centering
\caption{تنظیمات الگوریتم ژنتیک}
\label{tab:ga_params}
\begin{tabular}{ccc}
\hline
\textbf{پارامتر} & \textbf{مقدار} & \textbf{واحد} \\
\hline
اندازه جمعیت & $150$ & - \\
حداقل جمعیت & $100$ & - \\
حداکثر جمعیت & $200$ & - \\
تعداد نسل‌ها & $150$ & - \\
احتمال تقاطع & $0.7$ & - \\
احتمال جهش & $0.2$ & - \\
تحمل همگرایی & $0.1$ & - \\
ضریب تنکی & $0.02$ & - \\
مقیاس خطای درصدی & $1000.0$ & - \\
تعداد اجرای مقایسه‌ای & $10$ & - \\
\hline
\end{tabular}
\end{table}

