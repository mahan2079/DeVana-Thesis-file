
\chapter{مقدمه و مروری بر کار دیگران}

\section{مقدمه}

جاذب‌های دینامیکی ارتعاش
(\lr{DVAs})
 که در بسیاری از منابع مهندسی با عنوان
 جاذب‌های جرم تنظیم‌شده (\lr{TMD})
  نیز شناخته می‌شوند، ابزاری ساده اما مؤثر برای کاهش ارتعاشات در سازه‌ها هستند. این سامانه‌ها با استفاده از یک جرم کمکی که از طریق فنر و میراکننده به سازه متصل می‌شود، قادرند انرژی نوسانی را در فرکانس خاصی جذب کرده و شدت نوسانات را به‌طور چشمگیری کاهش دهند یا بازه ی فرکانسی آن را تغییر دهند. ایده اولیه این توسط  
 هرمان فرام
   در سال ۱۹۰۹ بازمی‌گردد 
   \cite{frahm1911device}
    و تحلیل نظری آن نخستین بار توسط 
اورموندرایود
     و
دن هارتوگ
در سال ۱۹۲۸ انجام شد 
 \cite{ormondroyd1928theory}.

از آن زمان تا کنون، جاذب‌های دینامیکی در بسیاری از حوزه‌های مهندسی جایگاه مهمی پیدا کرده‌اند. از کاربرد در موتورها و توربین‌ها گرفته تا سازه‌های بلندمرتبه، پل‌های طویل، کشتی‌ها و تجهیزات حساس، این سامانه‌ها به عنوان راه‌حلی مؤثر در کنترل ارتعاشات شناخته شده‌اند. در این فصل، تلاش شده است با نگاهی جامع به سیر تحول این فناوری، از نظریه‌های ابتدایی تا دستاوردهای روزآمد، تصویری کامل از وضعیت علمی و کاربردی جاذب‌های دینامیکی ارائه شود.

با پیشرفت فناوری و پیچیده‌تر شدن سازه‌ها و شرایط کاری، نیاز به طراحی‌های دقیق‌تر و سریع تر از گذشته بیشتر شده است. در نتیجه، جاذب‌های سنتی که عمدتاً به صورت خطی و غیرفعال طراحی می‌شدند، جای خود را به سامانه‌هایی با رفتار غیرخطی، چندمودی و حتی تطبیقی داده‌اند. توسعه جاذب‌های نیمه‌فعال و فعال، که می‌توانند با دریافت داده از حسگرها و اعمال پاسخ مناسب توسط عملگرها عملکرد خود را به‌طور بلادرنگ تنظیم کنند، گامی مهم در این مسیر بوده است. در این میان، ورود هوش مصنوعی و الگوریتم‌های یادگیری ماشین نیز افق‌های جدیدی را در طراحی، بهینه‌سازی و کنترل این سامانه‌ها گشوده است.

با وجود این پیشرفت‌ها، همچنان چالش‌هایی مانند نبود معیارهای استاندارد برای مقایسه سامانه‌ها، پیچیدگی مدل‌سازی رفتارهای غیرخطی، و فاصله میان تحقیقات آکادمیک و نیازهای واقعی صنعت وجود دارد. هدف از این فصل، ایجاد بستری برای شناخت جامع‌تر از قابلیت‌ها، محدودیت‌ها و مسیرهای آتی توسعه جاذب‌های دینامیکی است. ساختار این فصل به‌گونه‌ای تدوین شده است که ابتدا به معرفی مفاهیم پایه و سیر تاریخی این سامانه‌ها می‌پردازد، سپس به بررسی روش‌های طراحی کلاسیک، نوآوری‌های اخیر در جاذب‌های غیرخطی، و در نهایت به کاربردهای نوین در زمینه سامانه‌های فعال و هوشمند می‌پردازد.

% --- پایان بخش ---

\section{مبانی نظری جاذب‌های دینامیکی ارتعاش (دهه‌های ۱۹۲۰ تا ۱۹۵۰)}

ایده‌ی افزودن یک سامانه جرم-فنر به سازه اصلی به‌منظور حذف ارتعاشات ناخواسته، نخستین‌بار در سال ۱۹۰۹ توسط هرمان فرام ثبت اختراع شد \cite{frahm1911device}. این دستگاه که تحت عنوان جاذب ارتعاشی پویا معرفی شد، نخستین گام در مسیر توسعه‌ی کنترل ارتعاش غیرفعال به‌شمار می‌آید. با این‌حال، تا پایان دهه ۱۹۲۰ تحلیل نظری دقیق و منسجمی در این زمینه ارائه نشده بود. در سال ۱۹۲۸، اورموندرایود و دن هارتوگ مقاله‌ای با عنوان نظریه جاذب ارتعاشی پویا منتشر کردند که نخستین مدل‌سازی تحلیلی از جاذب ارتعاش بدون میرایی متصل به یک سامانه تک‌درجه‌آزادی (\lr{SDOF}) بدون میرایی را ارائه داد \cite{ormondroyd1928theory}. آن‌ها نشان دادند که اگر فرکانس جاذب دقیقاً با فرکانس طبیعی سازه اصلی تنظیم شود، می‌توان ارتعاش اصلی را در فرکانس تنظیم‌شده به‌طور کامل حذف کرد. البته این حذف کامل همراه با ایجاد دو تشدید جدید در فرکانس‌های مجاور خواهد بود. این نتیجه کلاسیک که اغلب در کتاب‌های درسی به آن پرداخته می‌شود، در شرایط بدون میرایی، پاسخ صفر در فرکانس تنظیم‌شده را پیش‌بینی می‌کند – پدیده‌ای که به‌روشنی در کتاب مشهور دن هارتوگ با عنوان ارتعاشات مکانیکی تشریح شده است \cite{den1985mechanical}.

پس از این پایه‌گذاری نظری، پژوهشگران دریافتند که حضور میرایی در جاذب، برای جلوگیری از پاسخ‌های نامتناهی در فرکانس‌های تشدید جدید، ضروری است. یکی از نقاط عطف در این زمینه، کار بیشاپ و ولبرن در سال ۱۹۵۲ بود که تحت عنوان مسئله جاذب ارتعاشی پویا در مجله \lr{Engineering (London)} منتشر شد \cite{bishop1952problem}. این پژوهش به بررسی اثر میرایی در عملکرد جاذب و سازه اصلی در کاربردهای مهندسی پرداخت. هم‌زمان، جی. ای. بروک نیز در سال ۱۹۴۶، تحلیلی مهم برای حالت وجود میرایی در جاذب و نبود آن در سازه اصلی ارائه کرد \cite{brock1946note}. وی رابطه‌ای تقریبی برای نسبت میرایی بهینه جاذب به‌دست آورد که با فرض نسبت جرم کوچک، پاسخ پیک سیستم را به حداقل می‌رساند. این رابطه که امروزه با نام \lr{Brock’s solution} شناخته می‌شود، نظریه نقاط ثابت \lr{Den Hartog} را با در نظر گرفتن میرایی توسعه داد و همچنان در طراحی‌های مهندسی کاربرد گسترده‌ای دارد. نتایج \lr{Brock} نشان دادند که افزودن میرایی مناسب به جاذب، گرچه نمی‌تواند ارتعاش را کاملاً حذف کند، اما می‌تواند پاسخ سیستم اصلی را به‌طور قابل‌توجهی کاهش دهد.

در نسخه‌های بعدی کتاب \lr{Den Hartog} نیز روش‌هایی برای تنظیم بهینه جاذب‌ها در حضور میرایی ارائه شده است \cite{den1985mechanical}. این روش که با عنوان «روش قله‌های برابر» (\lr{Equal-Peak Method}) شناخته می‌شود  و بعدها با نام روش \lr{$H_\infty$} نیز معرفی گردید هدفش تنظیم پارامترهای جاذب به‌گونه‌ای است که دو قله جدید در منحنی پاسخ، دارای دامنه‌ای برابر باشند. در پیکربندی کلاسیک یک سامانه جرم–فنر–میرای متصل به سازه‌ای بدون میرایی، \lr{Den Hartog} نسبت بهینه تنظیم فرکانس را به‌گونه‌ای استخراج کرد که این شرط برقرار شود. مجموع این دستاوردها از سوی \lr{Ormondroyd}، \lr{Den Hartog}، \lr{Bishop} و \lr{Brock} در نیمه نخست قرن بیستم، بنیان نظری طراحی و تحلیل جاذب‌های دینامیکی غیرفعال را شکل دادند.

در ادامه‌ی این مسیر، تلاش‌هایی نیز برای بررسی عملکرد جاذب‌های غیرخطی در شرایط ارتعاشات بزرگ صورت گرفت. از جمله پژوهش‌های اولیه در این زمینه، می‌توان به مطالعه دبلیو. جی. کارتر و اف. سی. لین در سال ۱۹۶۱ اشاره کرد که رفتار پایدار یک جاذب غیرخطی (با نیروی بازگرداننده غیرخطی) متصل به سازه اصلی را تحلیل کردند. این پژوهش که در مجله \lr{Journal of Applied Mechanics} منتشر شد، از نخستین مطالعاتی بود که بر محدودیت‌های نظریه خطی در ارتعاشات با دامنه بزرگ تأکید داشت  \cite{carter1961steady}. با این‌حال، تا دهه‌های بعد، عمده تمرکز تحقیقات همچنان بر سامانه‌های خطی و ارتعاشات با دامنه کوچک باقی ماند؛ چرا که تحلیل‌های ریاضی برای این نوع سیستم‌ها ساده‌تر بود.

به‌طور خلاصه، تا حدود سال ۱۹۶۰، نظریه پایه برای طراحی و تحلیل جاذب‌های دینامیکی خطی به‌خوبی تثبیت شده بود. نتایجی مانند اختراع فرام، نظریه تنظیم فرکانسی اورموندرایود و دن هارتوگ، روش قله‌های برابر دن هارتوگ، و رابطه میرایی بهینه بروک، پایه‌های اصلی طراحی جاذب‌های غیرفعال را شکل دادند. این روابط تحلیلی، امروزه نیز نقطه آغاز طراحی بسیاری از \lr{TMD}‌ها هستند و در مراحل بعد، با در نظر گرفتن پیچیدگی‌هایی مانند میرایی در سازه اصلی، مودهای متعدد یا رفتارهای غیرخطی اصلاح می‌شوند؛ موضوعاتی که در بخش ‌های بعدی به تفصیل بررسی خواهند شد.

% --- پایان بخش ---

\section{طراحی و بهینه‌سازی کلاسیک جاذب‌های دینامیکی ارتعاش (دهه ۱۹۶۰ تا ۲۰۰۰)}

پس از شکل‌گیری مبانی نظری اولیه جاذب‌های دینامیکی ارتعاش، پژوهش‌های دهه‌های بعد، به‌ویژه از دهه ۱۹۶۰ به بعد، عمدتاً بر بهینه‌سازی طراحی و توسعه مدل‌های قابل‌کاربردتر برای شرایط واقعی سازه‌ها متمرکز شدند. یکی از محورهای مهم تحقیقاتی در این دوران، بررسی اثر میرایی موجود در سازه اصلی بر عملکرد جاذب بود. در همین راستا، دبلیو. تامسون در سال ۱۹۸۱ با گسترش نتایج پیشین \lr{Brock}، فرمول‌هایی برای تنظیم بهینه فرکانس و میرایی جاذب در حضور میرایی ویسکوز در سازه اصلی ارائه داد که در شکل نمایش داده شده است \cite{thompson1981optimum}. این دستاورد از آن جهت اهمیت داشت که اغلب سازه‌های واقعی دارای میرایی ذاتی هستند و طراحی جاذب بدون درنظرگرفتن آن می‌تواند ناکارآمد باشد. \lr{Thomson} با ارائه نمودارها و روابط عملی، امکان تعیین پارامترهای بهینه جاذب برای کاهش پاسخ ارتعاشی سازه‌های میرادار را فراهم ساخت. در ادامه، او. نیشیهارا و تی. آسامی در سال ۲۰۰۲ این روابط را با ارائه پاسخ‌های دقیق تحلیلی برای سیستم‌های یک‌درجه‌آزادی دارای میرایی، توسعه دادند \cite{nishihara2002closed}. این مدل جدید که بر مبنای حل یک مسئله بهینه‌سازی \lr{$H_\infty$} بنا شده بود، دقتی بالاتر از راه‌حل تقریبی \lr{Den Hartog} در شرایط دارای میرایی ارائه می‌داد.

مسیر تحقیقاتی مهم دیگر در این دوره، گسترش کاربرد جاذب‌های دینامیکی از سیستم‌های ساده به سازه‌های چنددرجه‌آزادی (\lr{MDOF}) و پیوسته بود. در ساختارهایی مانند تیرها، ساختمان‌ها و ماشین‌آلات که دارای مودهای ارتعاشی متعدد هستند، انتخاب دقیق فرکانس‌های تنظیم‌شده اهمیت بالایی دارد. بررسی‌های اولیه، مانند کار \lr{Bishop} و \lr{Welbourn} در سال ۱۹۵۲، اهمیت اندرکنش مودال و لزوم تدوین استراتژی‌های تنظیم پیشرفته را مطرح کردند \cite{bishop1952problem}. در دهه‌های ۱۹۷۰ و ۱۹۸۰، روش‌هایی برای تنظیم جاذب‌ها به مودهای خاص در سازه‌های انعطاف‌پذیر توسعه یافتند. از جمله، فالکون و همکاران در سال ۱۹۶۷، روش‌هایی تحلیلی و گرافیکی برای بهینه‌سازی جاذب‌ها در سیستم‌های چندمودی ارائه دادند \cite{falcon1967optimization}.

در دهه ۱۹۹۰، پژوهشگرانی نظیر وای. کی. ون، تی. ایگوسا و کی. شو به‌صورت نظام‌مند استفاده از چندین جاذب جرم-تنظیم‌شده (\lr{MTMD}) را برای کنترل هم‌زمان مودهای مختلف بررسی کردند \cite{igusa1994vibration}. مطالعه‌ای کاربردی از کلارک در سال ۱۹۸۸ نشان داد که چگونه استفاده از چندین جاذب غیرفعال در یک ساختمان می‌تواند پاسخ لرزه‌ای ناشی از زلزله را کاهش دهد \cite{clark1988multiple}.

برتری اصلی جاذب‌های چندگانه در این است که می‌توانند چندین فرکانس تشدید را هم‌زمان هدف قرار دهند و نسبت به تغییرات خواص سازه‌ای، عملکرد پایدارتری ارائه دهند. این ویژگی به‌ویژه در برابر تغییرات ناخواسته یا عدم قطعیت در سیستم بسیار سودمند است. تا اوایل دهه ۲۰۰۰، طراحی و مکان‌یابی بهینه این جاذب‌ها به یک موضوع پژوهشی فعال تبدیل شد که در آن، تعیین آرایش مناسب چند جرم کوچک بر روی سازه به صورت یک مسئله بهینه‌سازی مدل‌سازی می‌شد. این پیچیدگی زمینه‌ساز بهره‌گیری از الگوریتم‌های محاسباتی و رویکردهای عددی برای حل مسائل طراحی شد و مقدمات توسعه تکنیک‌های مدرن را فراهم ساخت.

در این بازه زمانی، تلاش‌های متعددی برای اصلاح معیارهای بهینه‌سازی جاذب‌های غیرفعال صورت گرفت. برخی مطالعات به جای معیارهای سنتی، به سراغ کاهش پاسخ میانگین مربعی ارتعاشات تصادفی یا بهینه‌سازی بر مبنای هنجار \lr{$H_2$} رفتند تا به‌عنوان مکمل راه‌حل‌های \lr{$H_\infty$} موجود، ابزارهای تحلیلی دقیق‌تری ارائه دهند. همچنین در این دوره، ایده تنظیم جاذب به مودهای بالاتر سازه مطرح شد و روابط تقریبی مناسبی برای طراحی در این شرایط ارائه گردید. این دستاوردها مجموعه‌ای جامع از ابزارهای نظری و تحلیلی را برای طراحی جاذب‌های غیرفعال فراهم آوردند و بستر را برای ورود به نسل‌های جدیدتر جاذب‌ها و روش‌های کنترل هوشمند در قرن جدید آماده ساختند \cite{sadek1997method, tsai1993optimum, asami2002h, asami2002analytical}.

% --- پایان بخش ---

\section{پیشرفت‌های جاذب‌های غیرفعال ارتعاش (۲۰۰۰ تا ۲۰۲۵)}

\subsection{جاذب‌های پاندولی و مایع (گونه‌های مختلف \lr{TMD})}

در حالی‌که جاذب‌های دینامیکی کلاسیک معمولاً از یک جرم لغزنده متصل به فنر استفاده می‌کردند، مهندسان برای بهبود عملکرد و سهولت کاربرد، گونه‌های جایگزینی از این سیستم‌ها را توسعه دادند. یکی از این گونه‌ها، جاذب پاندولی تنظیم‌شده (\lr{Pendulum Tuned Mass Damper – PTMD}) است که به‌جای جرم لغزنده، از یک پاندول آویزان استفاده می‌کند. در این سیستم، نیروی بازگرداننده موردنیاز توسط نیروی گرانش حاصل از طول پاندول تأمین می‌شود. \lr{PTMD}ها به‌ویژه در سازه‌های بلند نظیر برج‌ها و دودکش‌ها کاربرد دارند، چرا که می‌توان فرکانس طبیعی سیستم را تنها با تغییر طول پاندول تنظیم کرد و این ویژگی فرآیند تنظیم مجدد پس از نصب را بسیار ساده‌تر می‌سازد \cite{saeed2023review, sun2018bi}.

مزیت دیگر این جاذب‌ها، توانایی نوسان در دو جهت افقی است که عملاً عملکردی همانند یک جاذب دوبعدی (\lr{Bi-directional TMD – BTMD}) فراهم می‌سازد؛ خصوصاً در سازه‌هایی که نیاز به کنترل ارتعاش در هر دو محور افقی دارند، مانند آسمان‌خراش‌ها یا دکل‌های مهاربندی‌شده. مطالعات نشان داده‌اند که برای زوایای نوسانی کوچک (کمتر از حدود ۵ تا ۱۰ درجه)، رفتار دینامیکی پاندول به‌صورت خطی بوده و عملکرد آن مشابه یک \lr{TMD} معادل است. در این راستا، \lr{Xu} و همکاران (۲۰۲۱) اثرات غیرخطی پاندول را در دامنه‌های بزرگ‌تر بررسی کردند و نتیجه گرفتند که در اغلب طراحی‌های واقعی، این اثرات قابل اغماض هستند و استفاده از نظریه خطی همچنان معتبر است.

تحقیقات نظری و تجربی متعددی پیرامون \lr{PTMD}ها صورت گرفته است. برای مثال، \lr{Vyas} و \lr{Bajaj} (۲۰۰۱) سامانه‌ای با چندین پاندول را به‌عنوان جاذب‌های خودپارامتری تحلیل کردند و \lr{Sun} و \lr{Jahangiri} (۲۰۱۸) یک جاذب پاندولی سه‌بعدی را برای برج‌های توربین بادی دریایی توسعه دادند که کاهش قابل‌توجهی در نوسانات طولی و عرضی ایجاد کرد \cite{saeed2023review}.

نوع دیگر از جاذب‌های جایگزین، جاذب‌های مایع تنظیم‌شده هستند که با نام‌های \lr{Tuned Liquid Damper (TLD)} یا \lr{Tuned Liquid Column Damper (TLCD)} شناخته می‌شوند. در این سیستم‌ها، به‌جای جرم جامد، از یک مایع (معمولاً آب) در یک مخزن یا لوله \lr{U}-شکل به‌عنوان جرم متحرک استفاده می‌شود. با تنظیم عمق مایع و هندسه مخزن، فرکانس نوسان مایع به فرکانس ارتعاش سازه تنظیم می‌گردد. این سامانه‌ها در برخی ساختمان‌ها و پل‌ها برای کنترل ارتعاشات ناشی از باد یا زلزله نصب شده‌اند \cite{lee2007real}. البته طراحی و مدل‌سازی آن‌ها به‌دلیل رفتار غیرخطی ناشی از حرکت موجی مایع، چالش‌برانگیز است. برای ساده‌سازی تحلیل، روش‌هایی مانند خطی‌سازی معادل برای تعیین میرایی مایع توسعه یافته‌اند.

در مجموع، جاذب‌های پاندولی و مایع با گسترش ابزارهای موجود در حوزه کنترل ارتعاش غیرفعال، راهکارهایی عملی برای کاربردهای خاص فراهم آورده‌اند. از جمله مزایای این سیستم‌ها می‌توان به امکان تنظیم آسان، عملکرد دوبعدی در پاندول‌ها، و استفاده از منابع موجود (مانند مخازن آب ساختمان‌ها به‌عنوان \lr{TLD}) اشاره کرد. اگرچه دینامیک این سیستم‌ها با جاذب‌های جرم-فنر سنتی تفاوت دارد، اما اغلب طراحی آن‌ها همچنان بر پایه همان روابط کلاسیک انجام می‌شود که برای شرایط خاص آن‌ها سازگار شده‌اند \cite{sun1995properties, zhang2022understanding, kashani2018numerical, frandsen2005numerical, zhang2015nonlinear}.

\subsection{جاذب‌های چندگانه و توزیع‌شده}

همان‌طور که پیش‌تر اشاره شد، استفاده از چندین جاذب کوچک به‌جای یک جاذب بزرگ، می‌تواند عملکرد بهتری در مواجهه با گستره وسیع‌تری از فرکانس‌ها و افزایش پایداری سامانه فراهم سازد. این ایده از دهه ۱۹۸۰ در میان مهندسان سازه مورد توجه قرار گرفت. در چارچوب \lr{Multiple Tuned Mass Dampers (MTMD)}، پژوهشگرانی نظیر \lr{Igusa} و \lr{Xu} (۱۹۹۴) با رویکرد تحلیلی، تأثیرات میرایی مودال ناشی از نصب جاذب‌های متعدد (اعم از هم‌فرکانس یا ناهم‌فرکانس) بر سیستم‌های چندمودی را بررسی کردند \cite{igusa1994vibration}. نتایج آن‌ها نشان داد که آرایش یکنواخت از جاذب‌هایی با فرکانس نزدیک به مود هدف، می‌تواند عملکردی مشابه یک \lr{TMD} بزرگ ارائه دهد، اما با حساسیت کمتر نسبت به خطای تنظیم.

در سازه‌های عمرانی، استفاده از چندین \lr{TMD} می‌تواند به‌معنای تخصیص یک جاذب به هر مود مهم باشد. برای مثال، \lr{Clarke} (۱۹۸۸) در یک پروژه عملی، چهار واحد \lr{TMD} را در یک ساختمان نصب کرد که هر یک به کنترل مود یا جهت خاصی اختصاص داشتند \cite{clark1988multiple}. در صنایع هوافضا و دریایی نیز از جاذب‌های توزیع‌شده بر طول تیرها یا بدنه کشتی‌ها استفاده می‌شود.

در مهندسی مکانیک، مفهوم جاذب‌های توزیع‌شده با ساختارهای متامتریال و تناوبی تلاقی می‌یابد. در این ساختارها، آرایه‌ای از رزوناتورها در طول یک سازه نصب می‌شوند که باعث ایجاد نوارهای فرکانسی (\lr{Bandgaps}) با میرایی شدید می‌گردد. این رویکرد در واقع توسعه پیوسته‌ای از ایده جاذب‌های چندگانه است. پژوهش‌های اخیر نشان داده‌اند که آرایش تناوبی از جاذب‌های \lr{DVA}-مانند می‌تواند از انتشار موج در تیرها جلوگیری کند \cite{nouh2014vibration, aladwani2020mechanics, pai2014acoustic, chavan2023programming}.

در مجموع، جاذب‌های چندگانه و توزیع‌شده نقش مهمی در بهبود پایداری و گستره فرکانسی کنترل ارتعاش دارند. تا دهه ۲۰۰۰، روش‌های طراحی مؤثری برای آن‌ها توسعه یافت؛ به‌ویژه روش‌هایی که با استفاده از بهینه‌سازی عددی، فرکانس و میرایی بهینه هر جاذب را تعیین می‌کردند \cite{zuo2005optimization, yang2015optimal, chun2015h, mayer2010approaches}. بررسی جامعی که در سال ۲۰۲۲ توسط \lr{Koutsoloukas}، \lr{Nikitas} و \lr{Aristidou} انجام شد، نشان داد که در میان ۲۰۸ مطالعه موردی از جاذب‌های جرم‌تنظیم‌شده واقعی، اگرچه بیشتر آن‌ها از یک \lr{TMD} منفرد استفاده می‌کنند، اما تعداد قابل‌توجهی از طرح‌ها به‌ویژه در پل‌های طویل و دال‌های بزرگ از جاذب‌های ترکیبی یا چندگانه بهره برده‌اند \cite{koutsoloukas2022passive}.

% --- پایان بخش ---

\subsection{جاذب‌های انرژی غیرخطی و سایر جاذب‌های غیرخطی}

یکی از تحولات مهم در دهه ۲۰۰۰، ظهور و توسعه جاذب‌های غیرخطی ارتعاش بود که با عنوان \lr{Nonlinear Energy Sinks (NES)} شناخته می‌شوند. برخلاف \lr{TMD}های کلاسیک که دارای فرکانس طبیعی مشخص و قابل تنظیم هستند، \lr{NES}ها معمولاً فاقد فرکانس تشدید خطی مشخص‌اند و در عوض، با بهره‌گیری از سختی غیرخطی (مثلاً به‌صورت تابع مکعبی)، می‌توانند انرژی ارتعاشی را از سیستم اصلی جذب کرده و در طیف وسیعی از فرکانس‌ها مستهلک کنند. ایده‌ی اصلی آن‌ها مبتنی بر «انتقال هدفمند انرژی» است؛ به این معنا که با تحریک سیستم اصلی، جاذب غیرخطی می‌تواند به‌طور غیرفعال انرژی ارتعاشی را به خود منتقل و سپس از طریق میرایی مستهلک کند.

مطالعه‌ی پیشگامانه واکاکیس و همکاران (۲۰۰۱) نشان داد که افزودن یک پیوستگی با رفتار غیرخطی قوی به یک سیستم خطی می‌تواند موجب انتقال یک‌طرفه انرژی از سازه اصلی به جاذب شود، بدون آنکه فرکانس خاصی برای تنظیم نیاز باشد \cite{vakakis2001inducing}. واژه \lr{Nonlinear Energy Sink} از همین قابلیت مشتق شده است؛ چرا که این جاذب می‌تواند انرژی را در گستره‌ای از فرکانس‌ها به دام انداخته و آن را به شکل موثری از بین ببرد. نمونه ساده‌ای از \lr{NES}، یک جرم متصل به فنر غیرخطی (مثلاً مکعبی) به همراه میرای است که با دریافت یک شوک از سازه، وارد رزونانس گذرا شده و انرژی ارتعاشی را جذب و از طریق میرایی تخلیه می‌کند.

از اواسط دهه ۲۰۰۰ به بعد، پژوهش در زمینه \lr{NES}ها رشد چشمگیری داشت. مرور جامعی توسط سعید، نصار و ال-شودیفات (۲۰۲۳) مجموعه‌ای از طراحی‌ها و کاربردهای متنوع این جاذب‌ها را گردآوری کرده است \cite{saeed2023review}. انواع مختلفی از \lr{NES}ها شامل نمونه‌های ضربه‌ای (\lr{Impact NES})، چرخشی (\lr{Rotary NES}) و دو‌حالت پایدار (\lr{Bi-stable NES}) تاکنون معرفی شده‌اند. جاذب‌های ضربه‌ای شامل جرمی هستند که درون محفظه حرکت کرده و با برخورد به موانع انرژی را از طریق ضربه‌های غیرکشسان مستهلک می‌کند \cite{wang2016numerical, li2025effectiveness, li2024irreversible}. در \lr{NES}های دو‌حالته، سازه دارای دو موقعیت تعادلی پایدار (مثلاً تیر کمانه‌شده یا نوسانگرهای مغناطیسی) است که می‌تواند به‌واسطه جهش دینامیکی، انرژی زیادی را جذب کند \cite{yao2020multi, zhang2017piezoelectric, bab2017vibration, jia2020review}. یائو و همکاران (۲۰۱۹) یک \lr{NES} سه‌حالته با سختی قطعه‌خطی طراحی کردند که توانایی بالایی در جذب شوک‌های شدید نشان داد \cite{yao2019tri}.

در کاربردهای مهندسی مکانیک، زوکا و افشرفرد (۲۰۱۹) یک جاذب غیرخطی با دو سختی طراحی کردند که همزمان به‌عنوان برداشت‌کننده انرژی نیز عمل می‌کرد و نشان دادند که در محدوده وسیع‌تری از فرکانس‌ها عملکرد بهتری نسبت به جاذب‌های خطی دارد \cite{zoka2019double}. در مهندسی عمران، ویرشام اثبات کرد که \lr{NES} می‌تواند در برابر تحریکات گذرا (مانند زلزله) با سرعت بالا انرژی را مستهلک کرده و از ساختار محافظت کند \cite{wierschem2014targeted, wierschem2014experimental}.

مزیت اصلی جاذب‌های غیرخطی نسبت به جاذب‌های خطی، پهنای باند وسیع‌تری است که در آن قادر به جذب مؤثر انرژی هستند. البته پیش‌بینی رفتار آن‌ها پیچیده‌تر بوده و اغلب نیازمند تحلیل‌های عددی یا روش‌های تقریبی پیشرفته مانند نظریه \lr{Slow Invariant Manifold} است \cite{habib2019impulsive, habib2021tracking}. از یافته‌های تئوریک مهم می‌توان به مطالعاتی اشاره کرد که نشان داده‌اند افزودن جاذب‌های سبک‌وزن با غیرخطی قوی می‌تواند میرایی مؤثر ساختار را افزایش دهد بدون آن‌که پیک تشدیدی جدید ایجاد کند \cite{georgiades2009passive, georgiades2007broadband}. به عنوان مثال، ساپسیس و همکاران (۲۰۱۲) نشان دادند که با شرایط خاص، جاذب‌های غیرخطی محلی می‌توانند هم‌زمان سختی و میرایی مؤثر سازه را افزایش دهند \cite{quinn2013effective}.

یکی از ویژگی‌های مهم \lr{NES}ها، عدم نیاز به تنظیم فرکانسی دقیق است. این جاذب‌ها با هر مود فعال در لحظه تعامل کرده و انرژی آن را جذب می‌کنند. در طراحی‌های مدرن، رابطه نیرو–جابجایی غیرخطی این جاذب‌ها به گونه‌ای تنظیم می‌شود که منحنی جذب انرژی دلخواه حاصل گردد. برخی از طرح‌ها حتی خاصیت تطبیقی ذاتی دارند؛ برای نمونه، یک جاذب پاندولی خودپارامتری ممکن است برای ارتعاشات کوچک به‌صورت خطی عمل کرده و مانند یک \lr{TMD} رفتار کند، اما برای تحریکات شدید به‌صورت غیرخطی وارد عمل شده و نقش یک \lr{NES} را ایفا نماید \cite{kecik2022nonlinear, vakakis2022nonlinear, raze2020multimodal}.

در جمع‌بندی، جاذب‌های انرژی غیرخطی و دیگر انواع جاذب‌های \lr{DVA} غیرخطی، نشان‌دهنده یک تغییر رویکرد بنیادین از جاذب‌های «تنظیم‌شده» به جاذب‌های «پهن‌باند غیرفعال» هستند. این سیستم‌ها به‌ویژه برای کنترل ارتعاشات گذرا یا غیرایستان (نظیر زلزله و ضربه) کارایی بالایی دارند. مطالعات متعددی از سال ۲۰۱۵ تا ۲۰۲۵ به توسعه طرح‌های نوآورانه \lr{NES} اختصاص یافته‌اند \cite{gzal2024enhanced, yang2022research, guo2025vibration, treacy2024irrational, lizunov2024comparison, al2021comparison}. اعتبارسنجی تجربی این جاذب‌ها در سامانه‌های واقعی همچنان موضوعی فعال در تحقیقات است که چالش‌هایی در تنظیم دقیق و اطمینان از پایداری عملکرد دارد. با این حال، ادبیات موجود \lr{NES} را به‌عنوان شاخه‌ای کلیدی از پژوهش‌های مرتبط با جاذب‌های غیرفعال تثبیت کرده است که قابلیت‌های آن‌ها را فراتر از محدودیت‌های نظریه‌های خطی گسترش می‌دهد \cite{wu2022improved, lu2024hybrid, geng2024state, parseh2015performance, zheng2023h, qiu2018theoretical, wu2022estimation}.

% --- پایان بخش ---

\subsection{جاذب‌های مبتنی بر \lr{Inerter} و عناصر با سختی منفی}

در دهه گذشته، توجه بسیاری به بهبود عملکرد جاذب‌های غیرفعال با استفاده از عناصر مکانیکی نوین مانند \lr{Inerter} و سامانه‌های با سختی منفی جلب شده است. \lr{Inerter} یک تجهیز دومقطبی است که نیروی مقاومتی متناسب با شتاب نسبی بین دو سر خود تولید می‌کند و در عمل رفتاری مشابه با یک تقویت‌کننده جرم دارد\cite{ghaedi2017invited}\cite{wei2025comparative}\cite{fahimi2025seismic}\cite{jiang2025optimal}\cite{smith2020inerter}. با افزودن \lr{Inerter} به یک جاذب دینامیکی ارتعاش، می‌توان کارایی جرم را به‌طور چشمگیری افزایش داد. این نوع جاذب که با عنوان \lr{Tuned Mass-Damper-Inerter (TMDI)} یا \lr{Inerter-Based Vibration Absorber (IVA)} شناخته می‌شود، قادر است با جرمی کمتر نسبت به \lr{TMD} سنتی، همان سطح از میرایی را ایجاد کند، زیرا \lr{Inerter} باعث ایجاد اینرسی ظاهری بیشتر می‌شود.

مطالعات متعددی از سال ۲۰۱۵ به بعد به بهینه‌سازی طراحی این جاذب‌ها پرداخته‌اند. ماریان و جیرالیس در سال ۲۰۱۴ برای اولین‌بار مفهوم \lr{TMDI} را برای سازه‌های ساختمانی پیشنهاد دادند و از طریق شبیه‌سازی نشان دادند که افزودن \lr{Inerter} (به‌صورت متصل به زمین یا بین طبقات) می‌تواند پاسخ لرزه‌ای سازه را به‌طور مؤثری کاهش دهد\cite{marian2014optimal}. به‌دنبال آن، آلوتا و فایلا در سال ۲۰۲۱ طراحی بهبودیافته‌ای برای جاذب‌های مبتنی بر \lr{Inerter} ارائه دادند و فرمول‌های تنظیم تحلیلی آن را نیز استخراج کردند\cite{alotta2021improved}. نتایج آن‌ها نشان داد که عملکرد جاذب در کاهش ارتعاش نسبت به \lr{TMD}های معمولی در دامنه فرکانسی گسترده‌تری بهبود یافته است. همچنین جانگید در سال ۲۰۲۱ به بررسی بهینه‌سازی جاذب‌های \lr{Inerter} برای سازه‌های با جداسازی پایه پرداخت و نشان داد که افزودن \lr{IVA} بهینه می‌تواند تغییر مکان پایه را به‌طور چشمگیری کاهش دهد\cite{jangid2021optimum}.

یکی از موضوعات کلیدی در این دسته از پژوهش‌ها، توسعه قوانین تنظیم کلاسیک مانند روش \lr{Den Hartog} برای جاذب‌هایی است که شامل \lr{Inerter} هستند. از آنجا که \lr{Inerter} پارامتر جدیدی به نام اینرتانس به مدل اضافه می‌کند که مستقل از جرم فیزیکی است، فرآیند بهینه‌سازی پیچیده‌تر اما غنی‌تر می‌شود. به‌عنوان نمونه، وانگ و همکارانش در سال ۲۰۱۹ سامانه‌ای را بررسی کردند که در آن، سختی منفی به‌صورت موازی با \lr{Inerter} در یک \lr{DVA} قرار گرفته بود و این عناصر به‌گونه‌ای بهینه‌سازی شدند که عملکرد جاذب در فرکانس‌های پایین بهبود یابد\cite{wang2019parameters}. سختی منفی (که معمولاً با فنرهای پیش‌فشرده یا عملگرهای مغناطیسی پیاده‌سازی می‌شود) می‌تواند بخشی از سختی کلی سامانه را خنثی کند و در نتیجه امکان تنظیم فرکانس طبیعی پایین‌تر بدون نیاز به جرم زیاد فراهم شود.

در همین راستا، لی و همکاران (۲۰۲۲) ترکیبی از \lr{Inerter} و عنصر سختی منفی را معرفی کردند که توانست پهنای باند جذب ارتعاش را به‌طور قابل‌توجهی افزایش دهد\cite{zheng2022energy}. پیکربندی نوآورانه دیگری که توسط سو و همکاران در سال ۲۰۲۳ ارائه شد، از مکانیزم اهرمی برای تقویت حرکت ورودی به \lr{Inerter} چرخشی بهره می‌برد. این طراحی، اثر «تقویت اینرسی» را ایجاد می‌کند؛ به این معنا که حرکت کوچک در انتهای اهرم منجر به چرخش زیاد چرخ‌طیار در \lr{Inerter} شده و اینرسی مؤثر سامانه چند برابر می‌شود. آزمایش‌های آن‌ها نشان داد که این پیکربندی در فرکانس‌های پایین عملکرد مناسبی در کاهش ارتعاش دارد\cite{su2023analytical}.

در کنار این‌ها، پژوهشگران به بررسی سامانه‌های اهرمی و چرخ‌دنده‌ای نیز پرداخته‌اند، مانند \lr{Toggle-Brace Inerter (TBI)} که از مکانیزم‌های لینکی برای بزرگ‌نمایی جابجایی و انتقال آن به \lr{Inerter} استفاده می‌کند. در مجموع، جاذب‌های مبتنی بر \lr{Inerter} را می‌توان نقطه اوج تکامل جاذب‌های غیرفعال دانست که با استفاده از مفاهیم سنتز شبکه‌های مکانیکی، بدون نیاز به منبع انرژی خارجی، عملکردی بهینه و مؤثر ارائه می‌دهند\cite{chowdhury2024critical}\cite{chen2019inerter}\cite{smith2020inerter}.

از دیگر تحولات قابل توجه در این زمینه، توسعه جاذب‌هایی با کارکرد دوگانه است که علاوه بر کنترل ارتعاش، توانایی برداشت انرژی را نیز دارند. هرچند هدف اصلی این سامانه‌ها کاهش ارتعاش است، اما با استفاده از سازوکارهای رزنانسی، می‌توان انرژی ارتعاش را نیز به برق تبدیل کرد. برای مثال، وانگ و همکاران در سال ۲۰۲۳ یک جاذب با سختی شبه‌صفر طراحی کردند که با استفاده از یک سامانه الکترومغناطیسی، انرژی حاصل از ارتعاشات فرکانس پایین را برداشت می‌کرد و هم‌زمان عملکرد مؤثری در جذب ارتعاش داشت\cite{wang2023dual}. در نمونه دیگری، \lr{Zoka} و \lr{Afsharfard} در سال ۲۰۱۹ جاذبی آزمایشگاهی با فنرهای دوجهته و یک ژنراتور الکتریکی توسعه دادند که توانست هم‌زمان کاهش ارتعاش و تولید برق را محقق سازد\cite{zoka2019double}.

طراحی‌های چندمنظوره این‌چنینی، در راستای رویکردهای مهندسی پایدار در حال رشد هستند، اما همواره باید موازنه بین جذب انرژی و اثربخشی در کنترل ارتعاش رعایت شود. زیرا برداشت انرژی، نوعی میرایی اضافه می‌کند که ممکن است بسته به شرایط، مفید یا مضر واقع شود. پژوهش‌ها نشان می‌دهند که با تنظیم صحیح برهم‌کنش الکترومکانیکی، می‌توان به حالتی دست یافت که هم ارتعاش کاهش یابد و هم انرژی مفید تولید شود\cite{qiu2018theoretical}\cite{huang2023towards}\cite{hasheminejad2024energy}\cite{yuan2018simultaneous}\cite{li2023bi}.

در پایان این بخش، می‌توان گفت که سامانه‌های \lr{DVA} غیرفعال همواره در مسیر نوآوری و بهبود قرار داشته‌اند. از جاذب‌های کلاسیک جرم-فنر گرفته تا نمونه‌های پاندولی، مایع‌مبنا، چندواحدی، غیرخطی و مبتنی بر \lr{Inerter}، هر یک در پاسخ به محدودیت‌های خاصی توسعه یافته‌اند. برای مثال، جاذب‌های غیرخطی برای رفع محدودیت باند فرکانسی طراحی شده‌اند و جاذب‌های \lr{Inerter} برای کاهش نیاز به جرم بالا. در بسیاری از کاربردهای امروزی، ترکیب این فناوری‌ها نیز دیده می‌شود؛ مانند یک جاذب پاندولی با \lr{Inerter} یا سامانه چندبخشی شامل واحدهای غیرخطی. این طراحی‌های پیشرفته در مرز بین سامانه‌های غیرفعال و نیمه‌فعال قرار می‌گیرند، که در بخش بعدی به‌طور مفصل مورد بررسی قرار خواهند گرفت.

% --- پایان بخش ---
\section{جاذب‌های ارتعاش فعال، نیمه‌فعال و هیبریدی}

با پیشرفت فناوری در دهه ۱۹۸۰، به‌ویژه در زمینه حسگرها و عملگرها، امکان استفاده از کنترل فعال در جاذب‌های دینامیکی ارتعاش فراهم شد. برخلاف جاذب‌های سنتی که عملکرد آن‌ها تنها بر پایه تنظیم اولیه عناصر مکانیکی استوار است، جاذب‌های فعال که با عنوان‌هایی چون \lr{Active Mass Damper (AMD)} یا \lr{Active Tuned Mass Damper (ATMD)} شناخته می‌شوند، از عملگرهای هیدرولیکی، پنوماتیکی یا الکترومغناطیسی برای اعمال نیرو به جرم کمکی استفاده می‌کنند. این قابلیت به جاذب اجازه می‌دهد به‌صورت بلادرنگ خود را با شرایط متغیر سازه وفق دهد و حتی در برابر تحریک‌های تصادفی یا چندفرکانسی عملکرد موثری داشته باشد.

یکی از نخستین مرورهای جامع در این زمینه توسط تی. تی. سونگ در سال ۱۹۸۸ ارائه شد \cite{chang1980structural} که در آن پتانسیل به‌کارگیری جاذب‌های فعال در مهندسی سازه بررسی شده بود. مدت کوتاهی بعد، پروژه‌هایی در مقیاس واقعی در ایالات متحده توسط سونگ، راینهورن و همکارانش اجرا شد \cite{soong1991full} که کارایی بالاتر سامانه‌های فعال نسبت به جاذب‌های غیرفعال را نشان داد. این سامانه‌ها معمولاً شامل یک جرم تنظیم‌شده همراه با عملگر و کنترل‌کننده بازخوردی مانند \lr{PID}، \lr{LQR} یا الگوریتم‌های پیشرفته‌تر مبتنی بر فیدبک حالت بودند \cite{yu2016active, ulusoy2021metaheuristic, casciati2012active, morales2015active}.

تا اواسط دهه ۹۰، پروژه‌هایی نظیر نصب جاذب فعال در ساختمان \lr{Kyobashi Seiwa} در توکیو \cite{christenson2002semiactive} به‌عنوان نمونه‌های موفق عملی از این رویکرد مطرح شدند. همچنین گزارش جامعی از سوی هاوسنر و همکاران در سال ۱۹۹۷ تحت عنوان "کنترل سازه‌ای: گذشته، حال و آینده" منتشر شد \cite{housner1997structural} که بیانگر گذار این سامانه‌ها از تئوری به اجرا بود. با این حال، هزینه و مصرف توان بالا در سامانه‌های کاملاً فعال، پژوهشگران را به سوی توسعه سامانه‌های هیبریدی و نیمه‌فعال سوق داد؛ ترکیبی از اجزای غیرفعال با مؤلفه‌های قابل کنترل که نیاز به انرژی کمتری دارند.

جاذب‌های هیبریدی ترکیبی از مزایای سامانه‌های غیرفعال و فعال را ارائه می‌دهند. ساختار پایه آن‌ها همانند \lr{TMD}های سنتی از جرم، فنر و میرای مکانیکی تشکیل شده، اما با افزودن یک عملگر کوچک یا کنترل‌کننده بازخوردی تقویت می‌شود. برخلاف \lr{AMD}های تمام‌فعال که برای عملکرد مؤثر به انرژی زیاد و عملگرهای قدرتمند نیاز دارند، جاذب‌های هیبریدی بیشتر اتکا به بخش غیرفعال دارند و بخش فعال صرفاً برای اصلاح عملکرد، تنظیم میرایی یا واکنش به شرایط پیش‌بینی‌نشده وارد عمل می‌شود. این هم‌افزایی باعث می‌شود عملکرد بالا با مصرف انرژی کمتر و قابلیت اطمینان بیشتر حاصل شود \cite{chey2007passive, stanikzai2022recent, rahimi2020application, chesne2019innovative, el2013recent}.

جاذب‌های نیمه‌فعال مانند \lr{Semi-Active Tuned Mass Damper (STMD)} جایگزینی بسیار بهینه از نظر مصرف انرژی برای جاذب‌های تمام‌فعال هستند. برخلاف سامانه‌های فعال که انرژی مکانیکی را مستقیماً وارد سازه می‌کنند، سامانه‌های نیمه‌فعال با تغییر در ویژگی‌هایی مانند میرایی یا سختی به‌صورت بلادرنگ و متناسب با پاسخ سازه عمل می‌کنند. این تغییرات معمولاً از طریق عملگرهای کم‌مصرف یا شیرهای کنترلی الکترونیکی انجام می‌شود که بدون تزریق انرژی زیاد، رفتار سیستم را تنظیم می‌کنند \cite{pastia2013vibration, kim2015control, yan2023seismic, demetriou2016novel, liedes2009improving}.

برای نمونه، برخی جاذب‌های نیمه‌فعال از سیالات \lr{Magnetorheological (MR)} یا شیرهای با دهانه متغیر بهره می‌گیرند که با دریافت سیگنال از حسگرها، مقاومت در برابر حرکت را به‌سرعت تنظیم می‌کنند \cite{eroglu2013observer, jenivssemi, masa2023review, peng2011development}. اگرچه این سامانه‌ها انرژی مستقیمی برای مقابله با حرکت وارد نمی‌کنند، اما توانایی تطبیق سریع آن‌ها باعث می‌شود عملکردی نزدیک به سامانه‌های تمام‌فعال داشته باشند. این ویژگی آن‌ها را برای کاربردهایی با محدودیت توان مانند زیرساخت‌های دورافتاده، ساختمان‌های بلند یا سازه‌های هوافضایی مناسب می‌سازد.

در همین راستا، ناگاراجایا و وارداراجان در سال ۲۰۰۵ مفهوم \lr{Variable Stiffness TMD} را معرفی کردند \cite{nagarajaiah2005short} که در آن سختی فنر طی عملکرد سیستم قابل تغییر است و بدین ترتیب امکان تنظیم مجدد فرکانس جاذب در مواجهه با تغییرات سازه‌ای فراهم می‌شود. پژوهش‌های بعدی \cite{nagarajaiah2009adaptive} نشان دادند که این رویکرد به‌ویژه در کنترل نیمه‌فعال بسیار کارآمد است، چراکه تغییر فرکانس تنظیم‌شده یکی از ضعف‌های اصلی جاذب‌های غیرفعال را برطرف می‌کند.

قوانین کنترلی برای جاذب‌های نیمه‌فعال می‌توانند از الگوریتم‌های ساده مانند کنترل \lr{Bang-Bang} (تغییر میرایی در نقاط اوج) گرفته تا روش‌های پیشرفته‌تری نظیر کنترل سختی متغیر بر پایه \lr{Lyapunov} یا کنترل فازی باشند \cite{ali2008semi, nguyen2018modeling, zhang2023seismic, kim2012semi, li2006fuzzy}.

از جمله روش‌های مؤثر، کنترل \lr{Skyhook} است که در آن میرای جاذب به‌گونه‌ای تنظیم می‌شود که گویی به نقطه‌ای ثابت در آسمان متصل شده و بدین ترتیب جذب انرژی بهینه می‌گردد. یاماگوچی و هارنپانچای در سال ۱۹۹۳ این روش را برای \lr{TMD}ها به‌کار بردند \cite{yamaguchi1993fundamental}. در سال‌های بعد، روش‌هایی مانند کنترل فازی برای جاذب‌های با سختی متغیر توسط علی و راماسوامی (۲۰۰۸) \cite{ali2009optimal} و مرور جامع کاربردهای کنترل فعال و نیمه‌فعال توسط کاسیاتی و همکاران (۲۰۱۲) \cite{casciati2012active} ارائه شد.

به‌طور خلاصه، گسترش دامنه جاذب‌ها به سوی سامانه‌های فعال و نیمه‌فعال، افق‌های جدیدی را در کنترل ارتعاش گشوده است. امروزه نقش الگوریتم‌های کنترلی در طراحی این سامانه‌ها، گاه به اندازه اجزای مکانیکی آن‌ها اهمیت دارد. در فصل بعد، چگونگی ورود هوش مصنوعی و یادگیری ماشین به این حوزه بررسی خواهد شد؛ جایی که مرز میان یک سامانه غیرفعال و یک سیستم کنترل هوشمند به‌تدریج ناپدید می‌شود.

% --- پایان بخش ---

\section{هوش مصنوعی و یادگیری ماشین در طراحی و بهینه‌سازی جاذب‌های دینامیکی ارتعاش (۲۰۱۵-۲۰۲۵)}

در دهه گذشته، ورود گسترده تکنیک‌های هوش مصنوعی به حوزه‌های مختلف مهندسی، مسیر تحقیقات در زمینه جاذب‌های ارتعاش را نیز دگرگون کرده است. این تأثیر در دو مسیر اصلی قابل مشاهده است: نخست، بهره‌گیری از الگوریتم‌های بهینه‌سازی مبتنی بر هوش مصنوعی برای تعیین پارامترهای بهینه جاذب‌ها؛ دوم، توسعه کنترل‌کننده‌های هوشمند که از شبکه‌های عصبی و یادگیری تقویتی برای کنترل بلادرنگ سیستم استفاده می‌کنند. در این فصل، هر دو رویکرد با استناد به منابع علمی منتشرشده طی سال‌های ۲۰۱۵ تا ۲۰۲۵ بررسی خواهند شد.

\subsection{بهینه‌سازی فراابتکاری پارامترهای جاذب ارتعاش}

فرآیند طراحی یک جاذب دینامیکی کارا، غالباً به حل یک مسئله پیچیده بهینه‌سازی منجر می‌شود؛ به‌ویژه در سیستم‌هایی با جاذب‌های چندگانه یا ویژگی‌های غیرخطی. روش‌های کلاسیک گاه قادر به یافتن راه‌حل‌های بهینه سراسری نیستند و در مینیمم‌های محلی گرفتار می‌شوند. در پاسخ به این چالش، پژوهشگران از الگوریتم‌های بهینه‌سازی فراابتکاری مانند الگوریتم ژنتیک (GA)، بهینه‌سازی گروه ذرات (PSO)، تکامل تفاضلی (DE)، تبرید شبیه‌سازی‌شده، الگوریتم خفاش و دیگر روش‌های الهام‌گرفته از طبیعت استفاده کرده‌اند \cite{chaudhary2021review}.

برای نمونه، الگوریتم ژنتیک (\lr{GA}) که مبتنی بر فرآیند انتخاب طبیعی است، از دهه ۲۰۰۰ برای طراحی \lr{TMD}ها به کار گرفته شده است. در این الگوریتم، فرکانس تنظیم و نسبت میرایی به‌صورت ژن تعریف می‌شوند و با تکامل جمعیتی از راه‌حل‌ها، کمینه‌سازی تابع هدف (مانند بیشینه دامنه پاسخ) انجام می‌گیرد \cite{mohebbi2013designing, colherinhas2019optimal, lavan2017multi, etedali2018optimum}. در بسیاری از پژوهش‌ها، از \lr{GA} برای بهینه‌سازی چندهدفه نیز بهره گرفته شده است.

الگوریتم \lr{PSO} به دلیل سادگی و سرعت همگرایی، از محبوبیت بالایی برخوردار است. این الگوریتم، با شبیه‌سازی رفتار اجتماعی پرندگان، به‌خوبی توانسته است پارامترهای جاذب‌های غیرخطی را بهینه کند \cite{balaji2021applications, shami2022particle}. برای مثال، از \lr{PSO} برای بهینه‌سازی توزیع سختی جاذب‌های چندگانه در یک پل یا طراحی جاذب‌های اهرمی با چندین پارامتر استفاده شده است. مطالعه‌ای توسط شمس‌الدین و همکاران (۲۰۲۴) نشان داد که \lr{PSO} در طراحی \lr{DVA} با \lr{Inerter} هیدرولیکی عملکرد بهتری نسبت به \lr{GA} دارد \cite{shamseldin2024ai}.

الگوریتم تکامل تفاضلی (DE) نیز در طراحی جاذب‌های با ساختار غیرخطی یا ابعاد بالا به کار رفته است \cite{ahmad2022differential}. این الگوریتم با استفاده از عملگرهای جهش و ترکیب، توانایی بالایی در جستجوی سطوح هدف پیچیده دارد و برای طراحی جاذب‌هایی مانند \lr{NES} بسیار مناسب است.

سایر الگوریتم‌ها نظیر تبرید شبیه‌سازی‌شده برای اجتناب از بهینه‌های محلی، کلونی زنبور عسل مصنوعی و الگوریتم حشره شب‌تاب برای تعیین محل نصب جاذب‌ها در ساختمان‌های بلند، و PSO خودتطبیقی برای بهبود همگرایی استفاده شده‌اند \cite{zhe2019application}.

الگوریتم‌های فراابتکاری به‌ویژه برای بهینه‌سازی چندهدفه کاربرد زیادی دارند. اهداف متعددی چون کمینه‌سازی جابه‌جایی و هم‌زمان کاهش جرم جاذب، یا افزایش عمر خستگی سازه، در طراحی مدرن مطرح‌اند. تکنیک‌هایی مانند NSGA-II و PSO چندهدفه توانایی تولید مجموعه‌های \lr{Pareto} از طراحی‌های بهینه را دارند. برای مثال، در یک مطالعه، طراحی جاذب با تقویت‌کننده هیدرولیکی بر پایه \lr{PSO} انجام شد و نتایج نشان دادند که این الگوریتم راه‌حل‌هایی متنوع و کارآمد تولید می‌کند.

به‌طور کلی، استفاده از الگوریتم‌های بهینه‌سازی مبتنی بر هوش مصنوعی به یک ابزار اصلی در طراحی جاذب‌ها تبدیل شده است. این روش‌ها امکان یافتن پیکربندی‌های بهینه را برای سیستم‌هایی با پیچیدگی بالا فراهم می‌سازند و نتایج آن‌ها معمولاً با شبیه‌سازی عددی یا آزمایش‌های تجربی اعتبارسنجی می‌شود.

% --- پایان بخش ---


\section{انگیزه تحقیق}

در مهندسی مکانیک، طراحی و بهینه‌سازی
جاذب های دینامیکی ارتعاشات (\lr{DVAs}) به‌عنوان ابزاری مؤثر برای کاهش ارتعاشات در سیستم‌ها و بهبود عملکرد سازه‌ها در برابر تحریکات خارجی شناخته می‌شود. به‌ویژه در سیستم‌هایی با درجات آزادی بالا (\lr{MDOF})، طراحی صحیح و بهینه‌سازی \lr{DVAs} به‌طور چشمگیری بر کارایی و دوام سازه‌ها تأثیر می‌گذارد. یکی از چالش‌های بزرگ در این حوزه، بهینه‌سازی دستگاه‌های جذب ارتعاش برای دستیابی به بهترین عملکرد در شرایط مختلف عملیاتی است. فرآیند طراحی بهینه معمولاً شامل انتخاب پارامترهای سیستم مانند نسبت جرم، نسبت تنظیم و میرایی است که باید در برابر محدودیت‌های مختلف از قبیل فضای طراحی و هزینه‌های محاسباتی بهینه شوند.

با توجه به پیچیدگی‌های بالا در محاسبه پارامترهای بهینه، به‌ویژه محاسبه تابع پاسخ فرکانسی (\lr{FRF}) برای هر ترکیب از پارامترها، محاسبات این سیستم‌ها اغلب هزینه‌بر و زمان‌بر هستند. این موضوع موجب می‌شود که محاسبات پیچیده و ارزیابی‌های فراوان، فرآیند طراحی را به‌طور چشمگیری کند کرده و نیاز به منابع محاسباتی زیادی داشته باشد.

در این میان، سؤال اصلی که این پایان‌نامه به دنبال پاسخ به آن است، یافتن مجموعه بهینه پارامترهای \lr{DVA} است که نیازهای طراحان را برآورده سازد. این سؤال فراتر از یک مسئله فنی ساده است؛ زیرا طراحان در شرایط واقعی با محدودیت‌های متنوعی از جمله محدودیت‌های مکانیکی، اقتصادی، زیست‌محیطی و عملکردی مواجه هستند. بنابراین، نیاز به یک رویکرد جامع وجود دارد که نه تنها پارامترهای فنی را بهینه کند، بلکه این بهینه‌سازی را در چارچوب نیازهای عملیاتی واقعی طراحان قرار دهد.

این پایان‌نامه با رویکردی نوین و جامع به این مسئله می‌پردازد. ساختار کلی پژوهش و دستاوردهای اصلی آن به شرح زیر است:

در فصل سوم، رویکردی مبتنی بر «جداسازی» (\lr{Decoupling Approach}) ارائه می‌شود که برای ایجاد یک کاتالوگ فراگیر (\lr{Meta Catalogue}) برای طراحان در زمینه طراحی پارامترهای \lr{DVA} به کار گرفته شده است. این رویکرد با جداسازی پارامترهای مختلف سیستم و تحلیل تأثیر هر یک به صورت مستقل، امکان ایجاد یک پایگاه داده جامع از طراحی‌های ممکن را فراهم می‌آورد. این کاتالوگ به طراحان اجازه می‌دهد تا بر اساس نیازهای خاص خود، ترکیب‌های مختلف پارامترها را بررسی کرده و بهترین گزینه را انتخاب نمایند.

فصل چهارم به بررسی معیارهای سنتی بهینه‌سازی می‌پردازد و نشان می‌دهد که این معیارها معمولاً بر یک نیاز خاص تمرکز می‌کنند و قادر به پوشش ترکیبی از نیازها نیستند. در این فصل، مفهوم «معیار تکین» (\lr{Singular Criteria}) معرفی می‌شود که با ترکیب هوشمندانه معیارهای مختلف، امکان ارزیابی جامع‌تر عملکرد \lr{DVA} را فراهم می‌آورد. این معیار جدید سپس با استفاده از الگوریتم ژنتیک (\lr{GA}) به صورت عددی بهینه‌سازی می‌شود و نتایج آن به صورت کاربردی در اختیار طراحان قرار می‌گیرد. این فصل با ارائه معیار تکین، ابزاری قدرتمند برای ارزیابی عملکرد \lr{DVA} در شرایط پیچیده عملیاتی به دست می‌دهد.

فصل پنجم نرم‌افزار نوآورانه «\lr{DeVana}» را معرفی می‌کند که توسط نویسندگان این پایان‌نامه توسعه یافته است. این نرم‌افزار اولین نرم‌افزار از نوع خود در جهان است که به صورت متن‌باز ارائه شده و برای تمامی اهداف ذکر شده و ویژگی‌های بسیار بیشتر طراحی شده است. \lr{DeVana} نه تنها تمامی روش‌های توسعه یافته در این پژوهش را پیاده‌سازی کرده، بلکه با رابط کاربری کاربرپسند و قابلیت‌های گسترده، زمین بازی نهایی برای طراحی \lr{DVA}ها محسوب می‌شود. این نرم‌افزار امکان شبیه‌سازی، بهینه‌سازی، تحلیل آماری و مقایسه طراحی‌های مختلف را با سهولت و کارایی بالا فراهم می‌آورد و ابزاری نهایی و پیشرفته برای طراحان فراهم می‌آورد که به آنها اجازه می‌دهد تا فرآیند طراحی \lr{DVAs} را به‌طور مؤثر و بهینه انجام دهند.

در این تحقیق، ما به‌دنبال معرفی رویکردی نوین برای طراحی و بهینه‌سازی \lr{DVAs} با استفاده از نرم‌افزار متن‌باز \lr{DeVana} هستیم. این نرم‌افزار به کاربران این امکان را می‌دهد که از الگوریتم‌های بهینه‌سازی پیشرفته و تطبیقی استفاده کنند و فرآیند طراحی \lr{DVAs} را به‌طور مؤثر و بهینه انجام دهند. در این نرم‌افزار، الگوریتم \lr{Genetic Algorithm} (\lr{GA}) با ویژگی‌های جدید و پیشرفته‌ای چون کنترل تطبیقی پارامترها و استفاده از مدل‌های پیش‌بینی برای کاهش هزینه‌های محاسباتی ارائه شده است که می‌تواند در بهینه‌سازی دستگاه‌های جذب ارتعاش به‌طور قابل توجهی مؤثر باشد.

این تحقیق در راستای رفع برخی چالش‌ها و محدودیت‌های موجود در زمینه طراحی \lr{DVAs} و بهبود کارایی الگوریتم‌های بهینه‌سازی برای این سیستم‌ها انجام شده است. نوآوری‌ها و ویژگی‌های پیشرفته‌ای که در این تحقیق ارائه می‌شود، می‌تواند گامی بزرگ در توسعه روش‌های بهینه‌سازی هوشمند در طراحی \lr{DVAs} باشد.




