%-- نحوه‌ی تعریف و نمایش قضایا و تعاریف و غیره
\newtheorem{theorem}{قضیه}[section]
\renewcommand{\thetheorem}{\thesection\lr{--}\arabic{theorem}}

\newtheorem{lemma}[theorem]{لم}
\newtheorem{gh}[theorem]{قرارداد}
\newtheorem{hads}[theorem]{حدس}
\newtheorem{pro}[theorem]{گزاره}
\newtheorem{result}[theorem]{نتیجه}
\newtheorem{cor}[theorem]{فرع}

\theoremstyle{definition}
\newtheorem{exam}[theorem]{مثال}
\newtheorem{defi}[theorem]{تعریف}
\newtheorem{notee}[theorem]{تذکر}
\newtheorem{note}[theorem]{نکته}
\newtheorem{algor}[theorem]{الگوریتم}
\newtheorem*{solve}{حل}

\theoremstyle{remark}
\newtheorem{remark}[theorem]{تبصره}
\newtheorem{tamrin}[theorem]{تمرین}
\newtheorem{yad}[theorem]{یادآوری}

%-- تبدیل واژه‌ی اثبات به برهان
\renewcommand\proofname{\bf برهان}
%-- دستوری برای تغییر نام کلمه «کتاب‌نامه» به «مراجع» 
\renewcommand{\bibname}{مراجع}

%----------------------------------------------
%-- Adds Qaher's commands, Some USEFUL COMMANDS
%----------------------------------------------
\newcommand{\op}{\operatorname}

\newcommand{\Mod}{\op{mod}}
\newcommand{\Image}{\op{Im}}
\newcommand{\Id}{\op{Id}}
\newcommand{\cl}{\op{cl}}
\newcommand{\innt}{\op{int}}

\newcommand{\Fix}{\op{Fix}}
\newcommand{\argmin}{\op{argmin}}
\newcommand{\Argmin}{\op{Argmin}}
\newcommand{\diag}{\op{diag}}
\newcommand{\conv}{\op{conv}}

\newcommand{\Z}{\mathbb{Z}}
\newcommand{\Q}{\mathbb{Q}}
\newcommand{\R}{\mathbb{R}}
\newcommand{\Ci}{\mathbb{C}}
\newcommand{\eN}{\mathbb{N}}

\newcommand{\A}{\mathcal{A}}
\newcommand{\I}{\mathcal{I}}

\newcommand{\rn}{\mathbb{R}\sar{n}}
\newcommand{\Rm}{\mathbb{R}\sar{m}}
\newcommand{\conda}{(\lr{A})\xspace}
\newcommand{\xmu}{x_\mu }
\newcommand{\umu}{u_\mu }
\newcommand{\pmu}{\phi_\mu }
\newcommand{\muk}{\mu_k}
\newcommand{\ml}{\mu_l}
\newcommand{\ubar}{\overline{u}}
\newcommand{\xbar}{\overline{x}}
\newcommand{\maj}[1][\xmu]{\{ #1 \}}
\newcommand{\dom}{\op{dom}}
\newcommand{\xs}{x\sar{*}}
\newcommand{\hx}{\hat{x}}
\newcommand{\hg}{\hat{g}}
\newcommand{\xst}{x\sar{*}(t)}
\newcommand{\im}[1][I]{#1\ben{-}}
\newcommand{\myquad}{\;}
\newcommand{\myst}{\text{\small به طوری که}}
\newcommand{\myow}{\text{\small در غیر این صورت}}

\newcommand*\sar[1]{^{#1}}
\newcommand*\ben[1]{_{#1}}
\newcommand*\tawan[1]{^{#1}}
\newcommand*\bnsar[2]{_{#1}^{#2}}
\newcommand*\overflesh[1]{\overset{\xrightarrow{\hspace*{.6cm}}}{#1}}

\newcommand{\ovrline}[1]{\overline{#1}}
\newcommand{\sarc}{\sar{\circ}}

%-- دستوری برای ایجاد «به پیمانه‌ی y»
\newcommand*\peymane[1]{\; ({\fontsize{9}{2.15} \selectfont \text{پیمانه‌ی #1}})}
%-- دستوری برای ایجاد «x به پیمانه‌ی y»
\newcommand*\bepeymane[2]{\; (\fontsize{9}{2.15} \selectfont \text{#1 به پیمانه‌ی #2})}

%-- دستورات سرعت‌بخش «قاهری» در فرمول‌نویسی ریاضی
\newcommand*\ta[3]{\mathop {#1} \limits_{#2}^{#3}}
\newcommand*\sumta[2]{\mathop {\sum} \limits_{#1}^{#2}}
\newcommand*\tsumta[2]{{\textstyle\mathop {\sum} \nolimits_{#1}^{#2}}}
\newcommand*\limto[2]{\mathop {\lim} \limits_{#1\to #2}}
\newcommand*\limdown[2]{\mathop {\lim} \limits_{#1\downarrow #2}}

\newcommand*\down[2]{\mathop {#1} \limits_{#2}}
\newcommand*\up[2]{\mathop {#1} \limits^{#2}}

%-- دستوراتی برای تعریف علامت باز و بسته‌ی کوتیشن مارک
\newcommand{\qut}{ "}
\newcommand{\quti}{`` }

%-- دستور پاورقی pa برای پاورقی‌های لاتین
\newcommand*\pa[1]{\hspace*{-.05em}\LTRfootnote{\lr{#1}}\hspace*{-.05em}}

\let\oldarraystretch\arraystretch