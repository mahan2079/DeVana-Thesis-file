% !TEX root=../main.tex
\chapter*{\AbstractHeadTitle}\thispagestyle{empty}

در مهندسی مکانیک، جاذب‌های دینامیکی ارتعاشات (\lr{Dynamic Vibration Absorbers - DVAs}) به‌عنوان ابزاری مؤثر برای کاهش ارتعاشات در سیستم‌های با درجات آزادی بالا شناخته می‌شوند. با این حال، چالش اصلی در طراحی این دستگاه‌ها، یافتن مجموعه بهینه پارامترهای سیستم است که نیازهای طراحان را در شرایط عملیاتی واقعی برآورده سازد. محاسبات پیچیده تابع پاسخ فرکانسی (\lr{FRF}) برای هر ترکیب پارامترها، فرآیند طراحی را هزینه‌بر و زمان‌بر می‌سازد و نیاز به منابع محاسباتی زیادی دارد.

این پایان‌نامه با رویکردی نوین به این مسئله می‌پردازد و سه نوآوری اصلی ارائه می‌دهد که هر کدام به نیازهای واقعی طراحان پاسخ می‌دهند. نخست، رویکرد جداسازی (\lr{Decoupling Approach}) پارامترهای سیستم که امکان ایجاد کاتالوگ فراگیر (\lr{Meta Catalogue}) پارامترهای \lr{DVA} را فراهم می‌آورد و طراحان را از انجام محاسبات تکراری رها می‌سازد. دوم، معرفی معیار تکین (\lr{Singular Criteria}) که معیارهای سنتی بهینه‌سازی را که تنها بر یک نیاز خاص تمرکز می‌کنند، با رویکردی جامع جایگزین می‌کند و امکان ارزیابی ترکیبی نیازهای طراحان را فراهم می‌آورد. سوم، توسعه نرم‌افزار نوآورانه \lr{DeVana} که اولین نرم‌افزار از نوع خود در جهان است و به صورت متن‌باز ارائه شده است.

نرم‌افزار \lr{DeVana} تمامی روش‌های توسعه‌یافته در این پژوهش را پیاده‌سازی کرده و زمین بازی نهایی برای طراحی \lr{DVAs} محسوب می‌شود. این نرم‌افزار با الگوریتم‌های بهینه‌سازی پیشرفته مانند الگوریتم ژنتیک تطبیقی، امکان شبیه‌سازی، بهینه‌سازی، تحلیل آماری و مقایسه طراحی‌های مختلف را با سهولت و کارایی بالا فراهم می‌آورد. نوآوری‌های ارائه‌شده در این پژوهش نه تنها محدودیت‌های موجود در طراحی \lr{DVAs} را رفع می‌کنند، بلکه گامی بزرگ در توسعه روش‌های بهینه‌سازی هوشمند در مهندسی مکانیک برمی‌دارند و ابزاری قدرتمند برای طراحان فراهم می‌آورند تا در شرایط پیچیده عملیاتی، تصمیم‌گیری بهینه‌تری داشته باشند.

