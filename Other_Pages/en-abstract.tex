% !TEX root=../main.tex
\chapter*{\AbstractHeadTitle}\thispagestyle{empty}

In mechanical engineering, dynamic vibration absorbers (\lr{DVAs}) are recognized as an effective tool for reducing vibrations in systems with high degrees of freedom. However, the main challenge in designing these devices lies in finding the optimal set of system parameters that meet designers’ needs under real operating conditions. The complex computations of the frequency response function (\lr{FRF}) for each parameter combination make the design process costly, time-consuming, and computationally demanding.

This thesis addresses this issue with a novel approach and presents three main innovations, each responding to the practical needs of designers. First, the \lr{Decoupling Approach} for system parameters, which enables the creation of a comprehensive \lr{Meta Catalogue} of \lr{DVA} parameters, freeing designers from repetitive calculations. Second, the introduction of the \lr{Singular Criteria}, which replaces traditional optimization criteria—focused only on a single requirement—with a holistic approach that allows for combined evaluation of designers’ needs. Third, the development of the innovative software \lr{DeVana}, the first of its kind in the world, released as open-source.

The \lr{DeVana} software implements all the methods developed in this research and serves as the ultimate playground for designing \lr{DVAs}. Equipped with advanced optimization algorithms such as the adaptive genetic algorithm, it facilitates simulation, optimization, statistical analysis, and comparison of different designs with high ease and efficiency. The innovations presented in this research not only overcome existing limitations in \lr{DVA} design but also represent a significant step forward in the development of intelligent optimization methods in mechanical engineering, providing a powerful tool for designers to make better decisions under complex operating conditions.
